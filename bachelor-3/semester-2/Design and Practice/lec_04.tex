\lecture{4}{Tue 28 Jan 2020 14:44}{Lab 3}
\section{Response Characteristics}

\begin{table}[H]
	\centering
	\caption{P Response Characteristics}
	\label{tab:p-data}
\end{table} 

\begin{table}[H]
	\centering
	\caption{P Response Characteristics}
	\label{tab:p-data}
\end{table} 

\begin{table}[H]
	\centering
	\caption{P Response Characteristics}
	\label{tab:p-data}
\end{table}
\section{Varying P, I, and D}

\subsection{P Value}
\subsection{I Value}
\subsection{D Value}

\section{Consistent Oscillations}

%plot of 
Why do higher P values increase the oscillation of the output?

What is happening to the location of the closed loop poles?

\section{Output with Only Integral Control}

What does this tell you about the behaviour of a system using only integral control?

\section{Plot w/ Desired Angle}
PID values: 
\begin{description}
	\item[P] 
	\item[I]	
	\item[D]
\end{description}
