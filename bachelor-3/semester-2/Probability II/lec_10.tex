\lecture{10}{Mon 27 Jan 2020 08:33}{Transformation of Multiple Random Variables (Guest Lecture)} %TODO: Finish these notes 

\begin{itemize}
	\item continuous multiple r.v's
	\item single random variables:
		\begin{align*}
			X \sim f_{X}, C &= \left\{ x: f\left( x \right) > 0 \right\}  \\
			&= \text{ "set of all possible values of X"} \\
			 &= supp\left\{ X \right\}  \\
		.\end{align*}
\end{itemize}

Consider: \begin{align*}
	h : C &\longrightarrow \R \\
	x &\longmapsto h (x) = Y
.\end{align*}
We are interested in finding the pdf of $Y$. Assume $h$ is invertible, with inverse function $g$. 
\[
	f_{Y}\left( y \right) = f_{X}\left( g\left( y \right)  \right)  \|\frac{d}{dy} g\left( y \right) \|
.\] where $y \in D = \left\{ h\left( x \right)  \mid x \in  C \right\} $. 

Multiple RV $X_{1} , \ldots , X_{n} \sim f_{X}\left( x_{1} , \ldots , x_{n} \right) $ is a joint pdf. 
\begin{align*}
	C &= \left\{ x \in  \R^{n} \mid f\left( x  \right) > 0 \right\}  \\
	h  &: C \to \R^{n}
.\end{align*}
Assume $h$ is continuously differentiable with continuously differentiable inverse $g$.

We are interested in finding the pdf of $Y = h\left( X \right) $. 
\begin{align*}
Y =  	h\left( x \right) &= \left(h_{1}\left( x_{1} , \ldots  , x_{n} \right)  , \ldots , h_{n}\left( x_{1} , \ldots , x_{n} \right) \right)\\
	Y_1 &=  h_{1}\left( x_{1} , \ldots  , x_{n} \right)\\
	    &\vdots\\
	  Y_{n} &= h_{n}\left( x_{1} , \ldots , x_{n} \right) \\
	  \implies X_1 &= g_1 \left( Y_{1} , \ldots , Y_{n} \right)  \\
		       &\vdots \\
		        X_{n} &= g_{n}\left( Y_{1} , \ldots , Y_{n} \right) 
  \end{align*}
  \begin{eg}
	  \begin{align*}
	  	Y_1 &= X_1 - X_2\\
	Y_2 &= X_1 + X_2\\
	h_1\left( x_1, x_2 \right) &= x_1 - x_2 \\
	h_2 \left( x_1, x_2  \right) &= x_1 + x_2  \\
	\implies X_1 &= \frac{Y_1 + Y_2}{2} = g_1\left( y_1, y_2 \right)  \\
	X_2 &= \frac{Y_2 - Y_1}{2} = g_2\left( y_1, y_2 \right) 
	  \end{align*}
  \end{eg}

\begin{theorem}
\begin{align*}
	D &= \left\{ h\left( x \right)  \mid x \in  C \right\} \\
	&= supp\left( Y \right) \\
	f_{Y}\left( y_{1} , \ldots , y_{n} \right) &= f_{X}\left( g_{1}\left( y_{1} , \ldots , y_{n} \right)  , \ldots , g_{n} \left( y_{1} , \ldots , y_{n} \right) \right)  \left| J_{g}\left( y_{1} , \ldots , y_{n} \right) \right| \\
	J_{g} &=  \text{ "Jacobian transformation of $g$"} \\
	&= \begin{bmatrix} 
	\frac{\partial g_1}{\partial y_1} & \ldots & \frac{\partial g_1}{\partial yn} \\
        \vdots & & \vdots \\
\frac{\partial g_{n}}{\partial y_1}  & \ldots & \frac{\partial g_{n}}{\partial y_{n}} \end{bmatrix} 
.\end{align*}
\begin{proof}
	Proof: Let $Y \sim f_{Y}$. 
	\[
		P\left( Y \in  B \right) = \int_{B}^{ } f_{Y} \left( y_{1} , \ldots , y_{n} \right)  dy_{1} , \ldots , dy_{n}, \quad B \subset \R^{n}
	.\] 
	Now, 
	\begin{align*}
		P\left( Y \in  B \right)  &= P\left( h\left( X \right) \in  B \right)  \\
					  &= P \left( X \in  g\left( B \right)  \right)  \\
					  &= \int \ldots \int_{ g\left( B \right) }^{ } f_{X}\left(x_{1} , \ldots , x_{n}\right)    dx_{1} , \ldots , dx_{n} \\
					  &= \int \ldots \int\displaylimits_{ g^{-1}\left(g\left( B \right)\right) = B}^{ } f_{X}\left( g_1 \left( y_{1} , \ldots , y_{n} \right)  , \ldots , g_{n}\left( y_{1} , \ldots , y_{n} \right)  \right)\left| J_{g}\left( y_{1} , \ldots , y_{n} \right) \right| dy_{1} , \ldots , dy_{n}  \\
	.\end{align*}
	by multivariate change of variable formula. 
\end{proof}
	
\end{theorem}

\begin{example}
	Solve the system where $X_1, X_2 \sim Exp\left( 1 \right)  $ are independent distributions
		\begin{align*}
			f_{X_i}\left( x_{i} \right) &= e^{-x_{i}}, \quad 0 < x_{i} < \infty \\
			f_{X_1, X_2}\left( x_1, x_2 \right) &= f_{X_1}\left( x_1 \right) f_{X_2}\left( x_2 \right)  \\
							    &= e^{-\left( x_1 + x_2 \right) }
		\end{align*}	
	and the system is defined by 
	\begin{align*}
		Y_1 &= X_1 - X_2  \\
		Y_2 &= X_1 + X_2 
	.\end{align*}
	Find the joint density of $\left( Y_1, Y_2 \right) $. 	
	\begin{enumerate}
		\item Find inverse of these distributions. 
			\begin{align*}
X_1 &= \frac{Y_1 + Y_2}{2} = g_1\left( y_1, y_2 \right)  \\
	X_2 &= \frac{Y_2 - Y_1}{2} = g_2\left( y_1, y_2 \right) 
			.\end{align*}
		\item Find the Jacobian:
			\begin{align*}
				J_{g}\left( y \right) &= \begin{bmatrix} 
				\frac{\partial g_1}{\partial y_1} & \frac{\partial g_1}{\partial y_2} \\
			\frac{\partial g_2}{\partial y_1} & \frac{\partial g_2}{\partial y_2} \end{bmatrix}  \\
							  &= \begin{bmatrix} \frac{1}{2} & \frac{1}{2} \\
							  -\frac{1}{2} & \frac{1}{2}\end{bmatrix}  \\
								       &= \left( \frac{1}{2} \right) \left( \frac{1}{2} \right)  - \left( \frac{1}{2} \right) \left( -\frac{1}{2} \right)  \\
								       &= \frac{1}{2}
			.\end{align*}
		\item Find the support of Y. 
			\begin{align*}
				& \quad0 < x_1 < \infty , 1 < x_2 < \infty \\
				\implies \quad & 0 < \frac{y_1 + y_2}{2} < \infty , 0 <  \frac{y_2 - y_1}{2} < \infty \\
				\implies \quad& -y_2 < y_1  , y_1 < y_2 , 0 < y_2  < \infty \\
				\implies \quad& -y_2 < y_1 < y_2, 0 < y_2 < \infty
			.\end{align*}
		Then
		\begin{align*}
			f_{Y}\left( y_{1} , \ldots , y_{n} \right)  &= e^{-\frac{\left( y_1 + y_2  \right)}{2} - \frac{\left( y_2 - y_1 \right) }{2} } \left( \frac{1}{2} \right) \\
			&= \frac{e ^{-y_2}}{2} 
		.\end{align*}
		for $y \in D = \left\{ \left( y_1, y_2  \right)  \mid -y_2 < y_1 < y_2, 0 < y_2 < \infty \right\} $
	\end{enumerate}
	We can also find the pdf of $Y_1$, ie, the marginal density. 
	\begin{align*}
		
	.\end{align*}
\end{example}

\begin{example}
	Suppose that $X = \left( X_{1} , \ldots , X_{n} \right) ^{T}$ has a joint pdf $f$, $A$ is an invertible matrix $\left( n \times  n \right) $. Find the pdf of Y. 
	\begin{align*}
		Y &= Ax \\
		h\left( X \right)  &=  AX \\
		\|\frac{\partial g\left( y \right) }{\partial y} \| &= \|\frac{\partial A^{-1}y}{\partial y}\| = \|A^{-1}\|  \\
		\frac{\partial By}{\partial y} &= B \\
		f_X\left( y \right) &= f_{X} \left( A^{-1}y \right) \|get\left( A^{-1} \right) \|\\
	.\end{align*}
	Sometimes we are only interested in a  single function of $X_{1} , \ldots , X_{n}$, say $Y_1 = h_1\left( X_1, X_2 \right) $. 
	\begin{note}
		\[
			I\left( c < y_2 < \infty \right) = \begin{cases}
				1 & 0 < y_2 < \infty\\
				0 & \text{otherwise}
			\end{cases}
		.\] 
	\end{note}

\end{example}
 
%% Finish these on your own time
