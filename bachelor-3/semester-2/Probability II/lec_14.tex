\lecture{14}{Tue 04 Feb 2020 10:32}{Marginal Distributions for Order Statistics}
Let $1 \le k \le  n$. Let us find $f_{k ,\ldots, n}$, the joint pdf of $X_{(1)} , \ldots , X_{(n)}$. So we integrate out $x_{1} , \ldots , x_{n-1}$. Start with $x_1$. 
\begin{align*}
	f_{2 ,\ldots,n }\left( x_{2} , \ldots , x_{n} \right) &= \int_{-\infty}^{x_2} n! f\left( x_1 \right) \ldots f\left( x_{n} \right) dx_1  \\
							      &= n! F\left( x_2 \right) f\left( x_2  \right) \ldots f\left( x_{n} \right) 
.\end{align*}
Next, integrate out x2: 
\begin{align*}
	f_{3, \ldots, n}\left( x_{3} , \ldots , x_{n} \right) &= \int_{-\infty}^{x_{3}} n! F\left( x_2 \right) f\left( x_2 \right) f\left( x_3 \right) \ldots f\left( x_{n} \right) dx_{2}  \\
							      &= n! \frac{F\left( x_3 \right) ^2}{2}f\left( x_3 \right) \ldots f\left( x_{n} \right)
.\end{align*}
The next integral over $x_3$ will be from $-\infty$ to $x_4$ and will give 
\[
	f_{4, \ldots, n}\left( x_{4} , \ldots , x_{n} \right) = n!\frac{1}{3!} F\left( x_4 \right) ^3 f\left( x_4 \right) \ldots f\left( x_{n} \right) 
.\] 
After $k - 1$ integrations we will get 
 \[
	 f_{k - 1}\left( x_{k} , \ldots , x_{n} \right) = \frac{n!}{\left( k - 1 \right) !}f\left( x_{k} \right) ^{k - 1} f\left( x_{k} \right) \ldots f \left( x_{n} \right) 
.\] 
Next, let $r>k$ be given and let up compute $f_{k , \ldots, r}\left( x_{k} , \ldots , x_{r} \right) $, $r\le n$. we want to integrate out $x_{n}, x_{n-1} , \ldots , x_{r}$, in that order, from $f_{k , \ldots, n}\left( x_{k} , \ldots , x_{n} \right)$. 

First, integrate over $x_{n}$ : 
\begin{align*}
	f_{k , \ldots, n -1}\left( x_{k} , \ldots , x_{n-1} \right) &= \frac{n!}{\left( k - 1 \right) !}\int_{x_{n-1}}^{\infty} f\left( x_{k} \right) \ldots f\left( x_{n} \right) dx_{n}  \\
								    &= \frac{n!}{\left( k - 1 \right) !} f\left( x_{k} \right) \ldots f\left( x_{n-1} \right) \left( 1 - F\left( x_{n - 1} \right)  \right) 
.\end{align*}

Next, integrating over $x_{n - 1}$ we get 
\begin{align*}
	f_{k, \ldots, n - 2}\left( x_{k} , \ldots , x_{n - 2} \right) &= \frac{n!}{\left( k - 1 \right) !} f\left( x_{k } \right) \ldots f \left( x _{n - 2} \right) \left[ - \frac{\left( 1 - F\left( x_{n -1} \right)  \right) ^2}{2} \right] ^{\infty}_{x _{n -2}}\\
								      &= \frac{n!}{\left( k - 1 \right) !} f\left( x_{k} \right) \ldots f \left( x_{n - 2} \right)  - \frac{\left( 1 - F\left( x_{n -2} \right)  \right) ^2}{2}
.\end{align*}
Integrating next over $x_{n - 2}$ (from $x_{n - 3}$ to $\infty$) gives 
\[
		f_{k, \ldots, n - 2}\left( x_{k} , \ldots , x_{n - 3} \right)=    \frac{n!}{\left( k - 1 \right) !} f\left( x_{k} \right) \ldots f \left( x_{n - 3} \right)  - \frac{\left( 1 - F\left( x_{n -3} \right)  \right) ^3}{3!}
.\] 
After integrating out $x_{n}, x_{n-1} , \ldots ,  x_{r+1}$ ($n - r$ integrations) we get 
\[
		f_{k, \ldots, r}\left( x_{k} , \ldots , x_{r} \right)=    \frac{n!}{\left( k - 1 \right) !} f\left( x_{k} \right) \ldots f \left( x_{n - 3} \right)  - \frac{\left( 1 - F\left( x_{n -3} \right)  \right) ^3}{3!}
.\] 

% TODO: Stuff 1 was wrong

Integrating out all factors except $x_{k}$ from the above then gives the marginal pdf of $X_{\left( k \right) }$ to be 
\[
	f_{k}\left( x_k \right) = \frac{n!}{\left( k - 1 \right) !} F\left( x_{k} \right) ^{k - 1}f\left( x_{k} \right) \frac{\left( 1 - F\left( x_{k} \right)  \right) ^{n - k}}{\left( n - k \right) !}
.\] 
Finally, suppose $k < r$ and $r \ge  k + 2$. Let us compute the joint pdf of $\left( X_{\left( k \right) , X_{\left( r \right) }} \right) $. Start with the joint pdf $f_{k , \ldots, r}\left( x_{k} , \ldots , x_{n} \right) $ in $\left( * \right) $. Start by integrating out $x_{k + 1}$ (from $x_k$ to $x_{x + 2}$). We get 
\begin{align*}
	f_{k, k + 2, \ldots, r }\left( x_k, x_{k + 2}, \ldots, x _{r} \right) &= \left( s_1 \right) \int_{x_k}^{x_{k + 2}} f\left( x_k \right) \ldots f\left( x_r \right) dx_{k + 1} \left( s_2 \right) \\ 
									      &= \left( ts \right) f\left( x_k \right) \left( F\left( x_{k + 2} \right) - f\left( x_k \right)  \right) > f\left( x _{k + 2} \right) \ldots f \left( x_{r} \right)  
.\end{align*}

After doing the integrations over $x_{k+1} , \ldots , x_{r-1}$ ($r - k - 1$ integrations) we get 
