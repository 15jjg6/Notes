\lecture{2}{Wed 08 Jan 2020 15:30}{General Time Systems}

\section{Time and set-valued functions of time}

\begin{itemize}
	\item a time-domain $\T$ 
	\item a set $X$		
\end{itemize}
\[
\implies	X^{\T} = \{f : \T \to X\}
.\] 

We shall also require " partial functions", denoted by 
\[
	X^{\T} = \{f : \T' \to X  \mid \T' \subseteq \T \text{ is a  subinterval} \}
.\] 
By a subinterval, we mean $\T' \in  \{(a,b), [a,b), \text{ etc}\}$. 

Why do we need partial functions? 

\begin{example}
	Consider, an input/output system:
	\begin{align*}
		U &= L'(\R; \R) \\
		Y &= C ^{0}(\R, \R)
	.\end{align*}
	Given $\mu \in  U$, the outputs corresponding to this are 
	\begin{align*}
		\eta: t &\longrightarrow \eta(t) \\
		t &\longmapsto \eta(t) = \eta^{2}(t) + \mu	
	.\end{align*}

	This is a differential equation, which we can solve. 
\begin{align*}
	\frac{dn}{dt}    &=  n^2 + \mu (t) ,  \\
	u_{0} &= u(t) = \text{constant} \\
	dn &= n^2 dt + u_{0} dt = (n^2 + u_{0}  ) dt \\
	\implies \int_{x(t_{0}}^{x(t)} \frac{dn}{n ^2 + 1} &= \int_{t_{0}}^{t}  dt\\
	\arctan (x(t)) - \arctan (x(t_{0})) &= t - t_{0} \\
	n(t) &= n(t_{0}) + tan(t - t_{0})
.\end{align*}
	
	\begin{conclusion}
		Even if $\mu$ is nice, it can happen that $n(t)$ blows up in finite-time. 
	\end{conclusion}
\end{example}

Moreover, the finite-time when solution blows up depends on $t_{0}$, $x(t_{0})$. Therefore the outputs are only partially defined. 

Hence, we need "partial functions".

\begin{definition}
	A general time-system is $(U, Y, \T, \U, \Y)$ such that 
	\begin{enumerate}
		\item $U$ is a set (the input values)
		\item $Y$ is a set (the output values)
		\item $\T$ is a time domain
		\item $\U \subseteq U^{(\T)}$ (the inputs)
		\item $\Y \subseteq Y^{(\T)}$ (the outputs)

		Thus inputs are things like $\mu : \T' \to U, \T' \subseteq \T$ is a subinterval. Similarly, outputs are things like $\eta : \T' \to Y$ where $\T' \subseteq \T$ is a subinterval. Points in the input set are $u \in  U$ and points in the output set are $y \in Y$. 
	\item $B \subseteq \U \times  \Y$ is such that, if $(\mu , \eta) \in  B$, then $\mu, \eta : \T' \to \T$, ie, $\mu$ and $\eta$ have the same domain. 

	\end{enumerate}
\end{definition}

\begin{notation}
	If $\xi \in X^{\T}$, ie, $\xi : \T \to X$. Suppose we are given a distinguished "starting time" $t_{0}  \in  \T$. If $t \ge t_{0}$ we have \[
	\xi_{[t_{0}, t)} = \xi  \mid [t_{0}, t)
	,\]  
	and \[
	\xi_{[t_{0}, t]} = \xi  \mid [t_{0}, t]
	.\] 
\end{notation}


If $(U, Y, \T, \U, \Y)$ is a general time system, we denote $B_{[t_{0}, t)}$ to be $(\mu _{[t_{0}, t)}, \eta_{[t_{0}, t)})$ for $(\mu , \eta) \in  B$. 

Similarly we have $B_{[t_{0}, t]}$. 

If $\mu \in \U$, a fixed input, then denote
\[
B(\mu ) _{[t_{0}, t)} = \xi\{(\mu _{[t_{0}, t)}, n _{[t_{0}, t)})  \mid (\mu , \eta) \in  B( \mu ) \}
.\] 


\begin{definition}
	Let $(U, Y, \T, \U, \Y)$ be a GTS (General Time System). It is:
	\begin{enumerate}
		\item causal from $t_{0} \in  \T$ if 
			\begin{align*}
				(\mu _{1})_{[t_{0}, t]} &= (\mu _{2})_{[t_{0}, t]} \\
	\implies	B(\mu _{1})_{[t_{0}, t]} &= B(\mu _{2})_{[t_{0}, t]} \forall  t \ge t_0
			.\end{align*}
		The behaviour from times $t_0 - t$ are determined by inputs between $t_0 - t$. 

		"The behaviours for times less than $t$ do not depend on knowing inputs for times larger than $t$ "

		\item strongly causal for $t_{0} \in  \T$ if 
	\begin{align*}
		(\mu _{1})_{[t_{0}, t)} &= (\mu _{2})_{[t_{0}, t)} \\
\implies		B(\mu _{1})_{[t_{0}, t)} &= B(\mu _{2})_{[t_{0}, t)}
	.\end{align*}
	\end{enumerate}
\end{definition}

\begin{definition}[Finitely Determined]
	A GTS $(U, Y, \T, \U, \Y)$ is finitely determined from $\tau \ge t_{0}$ if, for every input $\mu$ and for outputs $\eta_{1}, \eta_{2}$ satisfy $\left( \mu, \eta _{1} \right) , \left( \mu, \eta_{2} \right) \in B\left( \mu \}  \right) $ then if: 
	\begin{align*}
		\left( \eta_{1} \right) _{[t_0, \tau] } &= \left( \eta  _{2} \right)  _{\left[ t_0, \tau  \right] } \\
		\implies \left( \eta _{1} \right) _{\ge t_0} &= \left( \eta    _{2} \right)_{\ge t_0} 
	.\end{align*}
\end{definition}


\begin{definition}
	Stationarity is "shift-invariance". If we still have start time $t_0$, let $t\ge t_0$. We can shift the signal to the left by $t - t_0$ :
	 \[
		 \tau ^{*} _{t_0, t} \xi \left( s \right)  = \xi \left( s - \left( t - t_0 \right)  \right) 	
	.\] 
\end{definition}

\begin{definition}
	A GTS $\left( U, Y, \T, \U, \Y, B \right) $ is 
	\begin{enumerate}
		\item stationary if $\tau  ^* _{t_0, t} B \subseteq B$
		\item strongly stationary if $\tau  ^* _{t_0, t} B = B$
	\end{enumerate}
	\begin{itemize}
		\item Stationary $\implies$ behaviours are shift-invariant
		\item Strongly stationary $\implies$ shift-invariant and no behaviours are lost by shifting. 
	\end{itemize}
\end{definition}

\begin{remark}
	Shift-invariant implies that if you have a shifted system, you can shift it back and it would be the same. 
\end{remark}
