\lecture{3}{Fri 10 Jan 2020 14:28}{Back to Signals}

\begin{example}
	We take $U = L ^{1}\left( \R, \R \right) $, $Y = C ^{0}\left( \R, \R \right)$ and define behaviours by pairs $\left( \mu,  \eta  \right) $ satisfying 
	\begin{align*}
		\eta \left( t \right)  &=  \mu \left( t \right) \eta \left( t \right) ^2 \\
		\implies \frac{d\eta }{dt} &= \mu  \left( t \right) \eta ^2 \\
		\implies \frac{d\eta}{\eta ^2}  &= \mu \left( t \right) dt \\
		\implies \int_{\eta \left( t_0 \right) } ^{\eta \left( t \right) } \frac{d\eta}{\eta ^2}    &= \int_{t_0}^{t}   \mu \left( \tau  \right) d \tau \\
		\implies -\frac{1}{\eta }  |  ^{\eta \left( t \right)}  _{\eta (t_0)}	&= \int_{t_0}^{t} \mu  \left( \tau  \right) d \tau  \\
		\implies \frac{1}{\eta \left( t \right) } &= \frac{1}{\eta \left( t_0 \right) } - \int_{t_0}^{t} \mu \left( \tau  \right) d \tau   \\
		\eta \left( t \right)  &= \frac{1}{\frac{1}{\eta \left( t \right) }} \\
			&= \frac{1}{\eta \left( t_0 \right) } - \int_{t_0}^{t} \mu \left( \tau  \right) d \tau  \\
		&=  \frac{\eta \left( t_0 \right)}{1 - \eta  \left( t_0 \right)  \int_{t_0}^{t} \mu  \left( \tau  \right)  d \tau  }	
	.\end{align*}
	Given any $t_0, \eta  \left( t_0 \right) $ and for any $\epsilon \in  \R_{>0}$, we can choose an input $\mu $ so that $\lim_{t  \to t_0 + \epsilon}  \eta \left( t \right)  = \pm \infty \implies$ even when $\mu $ is defined on all of $\R$, the output $\eta $ may not be. This is why we must work with partial functions.
\end{example}

We will consider differential equations of the form 
\[
	\xi \left( t \right)  = \hat{F} \left( t, \xi \left( t \right)  \right) 
.\] 
where $\xi  : \T \to X \subseteq \R^{n}$ and $\hat{F} : \T \times  X  \to \R^{n}$. We will need some conditions on $\hat{F}$ to ensure existence and uniqueness of solutions. In particular, we will allow $t \longmapsto \hat{F}\left( t, x \right) $ to be only measurable. 

\section{Signals (Again)}

We work with continuous time signals on a continuous time domain $\T \subseteq \R$. For $p \in [1, \infty)$, denote by $L^{p}_{loc} \left( \T ; \F \right) $.

\begin{example}
	\begin{enumerate}
		\item $\T = (0, 1]$, $f\left( t \right)  = \frac{1}{t}$, $f \in  L^{p}\left( (0, 1], \R \right) $ for which $p$? For no $p$ ! 

		However, $f \in  L^{p}_{loc} \left( (0, 1], \R \right) $ for every $p \in  [1, \infty)$. Let $\mathbb{K} \subseteq (0, 1]$ be compact $\implies \mathbb{K} = \left[ a, b \right] $ for $0<a<b\le 1$. Thus, since f is continuous and $\mathbb{K}$ is compact:
			\[
				f  \mid \mathbb{K} \in  L^{p} \left( \mathbb{K}; \R \right) 
			.\] 

		\item Let $\T = \R $ and take 
			\[f\left( t \right)  =
				\begin{cases}
					\frac{1}{t} & t > 0 \\
					0 & t \le  0
			\end{cases}	 
			.\] 
			For which $p$'s is $f \in  L^{p}_{loc}\left( \R; \F \right) $? Indeed, if $\mathbb{K} \subseteq \R$ is compact and if $0 \in  int\left( \mathbb{K} \right) $, then $f  \mid  \mathbb{K} \not\in L^{p}\left( \mathbb{K}; \R \right) $.
	\end{enumerate}
\end{example}


In the spaces $L ^{p}_{loc}\left( \T; \F \right) $, we have a notion of convergence. 

\begin{definition}
	Let $\left( f_{j} \right) _{j \in  \Z_{>0}}$ be the sequence in $L^{p}_{loc}\left( \R; \R \right)$	given by: 

		\[
		graphs go here 
		.\] 

		Claim that $\left( f_{j} \right) _{j}$ converges to $f\left( t \right)  = 0$ in $L^{p}_{loc}\left( \R; \R \right) $. Let $\mathbb{K}\subseteq \R$ be a compact subinterval. Choose $N \in  \Z _{>0}$ be sufficiently large that $\mathbb{K} \in  \left[ -\left( N - 1 \right) , N - 1 \right] $. Then noting that:
		\[
			\left( insert another graph here  \right) 
		.\] 
		
		Thus, for every $j \ge  N$, $f_{j}  \mid \mathbb{K} = 0$. 
		$\implies \left( f_{j} \mid \mathbb{K} \right) _{j \in  \Z_{>0}}$ converges to $0 \mid \mathbb{K}$. 
\end{definition}

Similarly, we can consider $C^{0} \left( \T; \F \right) $ and talk about convergence. 

\begin{definition}
	A sequence $\left( f_{j} \right) _{j \in  \Z_{>0}}$ in $C ^{0}\left( \T; \F \right) $ converges to $f \in  C^{0}\left( \T; \F \right) $ if, for every compact interval $\mathbb{K} \subseteq \T$, $\left( f_{j} \mid \mathbb{K} \right) _{j \in  \Z_{>0}}$ converges to $f \mid \mathbb{K}$ in $C^{0} \left( \mathbb{K}; \F \right) $ using the $\infty$-norm.
\end{definition}
