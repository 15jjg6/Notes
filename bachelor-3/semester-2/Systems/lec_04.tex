\lecture{4}{Mon 13 Jan 2020 16:27}{Discrete Time Analogues}

Last time: Continuous-time signal spaces $C^{0}\left( \T;\F \right) $ and $L^{p}_{loc}\left( \T; \F \right) $. 

Discrete-time analogs: Let $\T \subseteq \Z\left( \Delta \right) $ be a discrete time-domain. Denote: 
\[
	l_{loc}\left( \T; \F \right)   = \F ^{\T}
.\] 

\begin{definition}[Convergence]	
	 A sequence $\left( f_{j} \right) _{j \in  \Z_{>0}}$ in $l _{loc}\left( \T;\F \right) $ converges to $f$ in $l_{loc}\left( \T; \F \right)$ if, for every finite $\mathbb{K} \subseteq \T$, the sequence $\left( f_{j}  \mid \mathbb{K} \right) _{j \in  \Z_{>0}}$ converges to $f  \mid \mathbb{K}$. 

\end{definition}

Convergence in $l_{loc}\left( \T; \F \right) $ is really just pointwise convergence. 
\begin{note}
	 In all of the above cases, convergence is not, generally, norm convergence. It is norm convergence when $\T$ is compact (in the continuous case) or finite (in the discrete time case). The convergence is with respect to semi norms. The semi norms are:
	\begin{itemize}
		\item $L_{loc}^{p} \left( \T; \F \right) $ : $\norm{f}_{\mathbb{K}, p} = \|f \mid \mathbb{K}\|, \mathbb{K} \subseteq \T$ compact 
		\item $C^{0} \left( \T; \F \right) $: $\|f\|_{\mathbb{K}, \infty} =  \|f \mid \mathbb{K}\| _{\infty}, \mathbb{K}\subseteq\T$ compact
		\item $l _{loc}\left( \T;\F \right) $: $\|f\|_{\mathbb{K}, p} = \|f \mid \mathbb{K}\|_{p}, \mathbb{K}\subseteq \T$ finite, and convergence is independent of $p$.
	\end{itemize}
\end{note}

Convergence in these spaces in equivalent to convergence in each of the seminorms. One can also talk about continuity of functions to or from these spaces by $f$ is continuous iff $\lim_{j \to \infty} f\left( x_{j} \right) = f\left( x \right) \forall \left( x_{j} \right) _{j \in \Z_{>0}}$ converging to x. 

\section{Differential and Difference Equations (Ordinary)}

In your past life: 
\[
	\dot{x}  = F(t, x)
.\] 
This is not really a differential equation.

%% Todo: fix xdot and | switch to double bar  | |

\begin{definition}[Differential as a Mapping]
	Let $X \subseteq \R^{n}$ be open and let $\T\subseteq \R$ be a continuous time-domain. A differential equation with state space $X$ and time-domain $\T$ is a mapping $\hat{F}: \T \times X \to \R^{n}$. 
\end{definition}

\begin{definition}[Absolute Continuity]	
	A mapping $f : \T \to \R$ is locally absolutely continuous if there exists $t \in \T$ and $g \in  L^{1}_{loc}\left( \T; \R \right) $ such that 
	\[
		f\left( t \right) = f\left( t_0 \right) + \int_{t_0}^{t} g\left( \tau  \right) d\tau  
	.\] 

	Properties of $f : \T \to \R$ :
	\begin{enumerate}
		\item $f$ is differentiable almost everywhere
		\item $f\left( t \right)   = g\left( t \right) $ for almost very $t \in  \R$.
	\end{enumerate}
\end{definition}

\begin{definition}
	Let $\hat{F} : \T \times X \to \R ^{n}$ be a differential equation. A solution to $\hat{F}$ is an absolutely continuous $\xi  : \T' \to X$ where $\T' \subseteq \T$ is a subinterval and where 
	\[
		\dot{\xi } \left( t \right)  = \hat{F}\left( t, \xi \left( t \right)  \right) 
	.\] 
	for almost every $t \in  \T'$. 
\end{definition}

\subsection{Conditions on $\hat{F}$ to ensure existence and uniqueness of solutions} 

\begin{definition}[Locally Lipschitz]
	Let $X \subseteq \R^{n}$ be open and let $f : X \to \R^{m}$. Say that $f$ is Lipschitz if $\exists    L \in  \R_{>0}$ (called a Lipshitz constant) such that $|f\left( x_1 \right)  - f\left( x_2 \right) | \le L|x_1 - x_2|$. Say that $f$ is locally Lipshitz if, for every compact $K \subseteq X$, $f|K$ is Lipshitz. 
\end{definition}

\begin{property}
	\begin{enumerate}
		\item If $f$ is locally Lipshitz, it is continuous. 
		\item If $f$ is continuously differentiable, it is locally Lipschitz. 
	\end{enumerate}
\end{property}

\begin{example}
	\begin{enumerate}
		\item $f : \R \to \R$ is $f\left( x \right)  = \sqrt{|x|} $. 

			%%Insert graph here
			In this example, $f$ is continuous but not locally Lipschitz. 
		\item $f : \R \to \R$, $f\left( x \right)  = |x|$. 
			%%Insert graph here (absolute value fun)

			We have:
			\[
				|f\left( x_1 \right) - f\left( x_2 \right) | \le  |x_1 - x_2|
			.\] 
			$f$ is locally Lipschitz, but is not continuously differentiable. 
	\end{enumerate}
\end{example}
