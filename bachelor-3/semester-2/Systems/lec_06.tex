\lecture{6}{Fri 17 Jan 2020 15:55}{Autonomous Difference Equations}

Last time: $\hat{F} : \T \times X  \to \R^{n}$ 

Solution: $\dot{\xi} \left( t \right) = \hat{F}\left( t, \xi \left( t \right)  \right) $, almost everywhere $t \in  \T' \subseteq \T$. 

Domain:
\[
D_{\hat{F}} = \left\{ \left( t, t_0, x_0 \right) \in \T\times \T\times X \\
\mid \text{ solution exists at time }t, w \text{ initial conclusion }x_0 \text{ at initial time } t_0\right\} 
.\] 

Flow: $\Phi ^{\hat{F}} : D_{\hat{F}} \to X$ satisfies 
\[
	\frac{d}{dt}\Phi ^{\hat{F}}\left( t, t_0, x_0 \right)  = \hat{F}\left( t, \Phi ^{\hat{F}}\left( t, t_0, x_0 \right)  \right) , \quad \Phi ^{\hat{F}}\left( t_0, t_0, x_0 \right)  = x_0
.\] 
\begin{example}
	$\T = \R$, $X = \R$, $\hat{F}\left( t, x \right) = x^2$. Solution with initial condition $x_0$ at time $t_0$ satisfies 
	\begin{align*}
		\dot{\xi }\left( t \right)  &=  \xi \left( t \right) ^2 \\
		\xi \left( t_0 \right) &= x_0 \\
		\dot{x} &= x^2 \\
		\frac{dx}{x^2} &= dt \\
		\implies x\left( t \right) &= \frac{x_0}{1 - x_0\left( t - t_0 \right) } 
	.\end{align*}

	\[
		D_{\hat{F}} = \left\{ \left( t, t_0, x_0 \right)  \mid \begin{cases}
				t \in \left( -\infty, t_0 + \frac{1}{x_0} \right) & x_0 > 0 \\
				t \in  \left( t_0 + \frac{1}{x_0}, \infty \right) & x_0 < 0 \\
				t \in  \left( -\infty, \infty \right)  & x = 0
		\end{cases} \right\} 
	.\] 
	\begin{align*}
		\Phi ^{\hat{F}}: D_{\hat{F}} &\longrightarrow \R \\
		\left( t, t_0, x_0 \right)  &\longmapsto \Phi ^{\hat{F}}(\left( t, t_0, x_0 \right) ) = \frac{x_0}{1 - x_0\left( t - t_0 \right) }
	.\end{align*}
\end{example}

For linear equations, there are no restrictions on the times for which solutions exist.
\[
\implies D_{\hat{F}} = \T \times  \T \times  X 
.\] 

\subsection{Autonomous ODE's (independent of time)}

An ODE $\hat{F} : \T \times  X  \to \R ^{n} $ is autonomous if $\exists  \hat{F}_{0} : X  \to \R^{n}$ so that $\hat{F}\left( t, x \right) = \hat{F} _{0}\left( x \right) $. This just captures the idea of independence of time. 

The flow of an autonomous ODE has the following property:
\[
	\Phi ^{\hat{F}}\left( t, t_0, x_0 \right)  = \psi^{\hat{F}}\left( t - t_1, x_0 \right) 
.\] 
for some $\psi^{\hat{F}} : \T \times  X \to X$. 

For autonomous equations, one often gives initial conditions at $t_0 = 0 $. 

\subsection{Autonomous Linear ODE's}

\begin{itemize}
	\item Linear means $\hat{F}\left( t, x \right)  = A\left( t \right) \cdot x$. Therefore linear and autonomous means: 
		 \[
			 \hat{F}\left( t, x \right)  = Ax \quad \text{for} \quad A \in  L\left( X;X \right) 
		.\] 
	\item For arbitrary (ie not autonomous) linear ODE's
		 \[
			 \Phi^{\hat{F}}\left( t, t_0, x_0 \right)  = \Phi_{A}\left( t, t_0 \right) \cdot x_0
		.\] 
		where $\Phi_{A} : \T \times \T \to L\left( X, X \right) $ is the state transition map. 

	\item In the autonomous case we can calculate the state transition map in terms of the operator exponential of $A in L\left( X, X \right) $. 

		First, if $L \in  L\left( X, X \right) $ 
		\[
		e^{L} = I + \sum_{n=1}^{\infty} \frac{L^{n}}{n!} 
		.\] 

	\item Consider the initial value problem
		\begin{align*}
			\dot{\xi}\left( t \right) &= A \xi\left( t \right)  \\
			\xi\left( t_0 \right) &= x_0 \\
			\implies \text{Solution is } \xi\left( t \right)  &= e^{A\left( t - t_0 \right) }x_0
		.\end{align*}
		Thus $\Phi_{A}\left( t, t_0 \right) = e^{A\left( t - t_0 \right) }$ and $\Phi^{\hat{F}}\left( t, t_0, x_0 \right) = e^{A\left( t - t_0 \right) }\cdot x_0$. 

	\item For linear ODE's autonomous means constant coefficients. 
\end{itemize}

\section{Difference Equations}

\begin{definition}
	Let $\T \subseteq \Z\left( \Delta\right) $ be a discrete time-domain and let $X \subseteq \R^{n}$ be open. A difference equation with time-domain $\T$ and state space $X$ is a mapping $\hat{F} : \T \times  X \to X$. A solution is a mapping $\xi : \T  \to  X $ satisfying
	\[
		\xi\left( t + \Delta \right)  = \hat{F} \left( t, \xi\left( t \right)  \right) 
	.\] 
\end{definition}

If we have an initial condition $\xi\left( t_0 \right)  = x_0$, then $\xi\left( t_0 + \Delta \right)  = \hat{F}\left( t, x_0 \right) $, $\xi\left( t_0 + 2 \Delta \right)  = \hat{F}\left( t, \hat{F}\left( t, x_0 \right)  \right)$, $\ldots$

What is $\xi\left( t_0 - \Delta \right) $? The final equation should satisfy 
 \[
	 \xi\left( t_0 \right)  = \hat{F}\left( t, \xi\left( t_0 - \Delta \right)  \right) 
.\] 
Generally, this expression cannot be saved for $\xi\left( t_0 - \Delta \right) $. Difference equations are meant to "Go forward". If we can go backwards we say the system is invertible. 

\begin{example}
	$\T = \Z$, $X = \R$, and $\hat{F}\left( t, x \right)  = 0$.
	\[
		\implies \xi\left( t + k\Delta \right)  = 0 \quad \forall  k \in \Z_{>0}
	.\] 
	With $\xi$ we do not have uniqueness of solutions. 

	Nevertheless we define flows of difference equations, by 
	\[
		\Phi^{\hat{F}}\left\{ \left( t, t_0, x_0 \right) \in  \T \times  \T \times  X  \mid  t \ge t_0 \right\} \to X
	.\] 
	defined by $\Phi^{\hat{F}}\left( t + \Delta, t_0, x_0 \right) =\hat{F}\left( t, \Phi^{\hat{F}}\left( t, t_0, x_0 \right)  \right) $, $\Phi^{\hat{F}}\left( t_0, t_0, x_0 \right) = x_0$.
	
	We can talk about linear and autonomous difference equations, and the statements for differential equations have analogues for difference equations. 
\end{example}
