\documentclass[a4paper]{article}

\usepackage[utf8]{inputenc}
\usepackage[T1]{fontenc}
\usepackage{textcomp}
\usepackage{amsmath, amssymb}


% figure support
\usepackage{import}
\usepackage{xifthen}
\pdfminorversion=7
\usepackage{pdfpages}
\usepackage{transparent}
\newcommand{\incfig}[1]{%
	\def\svgwidth{\columnwidth}
	\import{./figures/}{#1.pdf_tex}
}

\pdfsuppresswarningpagegroup=1

\usepackage{fancyhdr}
	\fancyhead{}  		% Clears all page headers and footers
	\pagestyle{fancy}
	\lhead{MTHE 335} 	% TODO: Add automatic context switching, auto complete current class
	\rhead{Joe Grosso}
	\chead{Lesson 1}	% TODO: Smart class auto complete
	\cfoot{ 2020-01-06 }

\newtheorem{define}{Definition}


\begin{document}
	
	\section{Introduction}
	\subsection{Marking Scheme}
	\begin{description}
		\item[Quizzes] 40\% in tutorial in even weeks, drop lowest score
		\item[Final Exam] 60\%. If final exam grade is <40\%, grade $=min\{final, usual\}$
	\end{description}
	
	Content of quiz 1:

	\begin{itemize}
		\item system of linear homogeneous equations 
		\item Scalar linear inhomogeneous differential equations 
	\end{itemize}
	

	\subsection{Course Content}
	This is a system theory course. 

	\subsection{General System Theory}
	
	\begin{define}
		A general input-output system is a triple $(u, y, B)$, where
	\end{define} 

	\begin{enumerate}
		\item $u$ is a set (the set of inputs)
		\item $y$ is a set (the set of outputs)
		\item $B \subseteq u X y$ (the behaviours).
	\end{enumerate}



	
	An element of $B$ looks like $(u, \mathbb{n}$ with $\mathbb{u}$ an input and $\mathbb{n}$ an output. If $\mathbb{u} \in \mathbb{U}$ we denote 
	\[
		B(u) = \{(\mathbb{u}, \mathbb{n} \in B\}
	.\]
	This is the set of all behaviors corresponding to $\mathbb{u}\in \mathbb{U}$. 

	Def: A functional input/output system is a general input/output system $(u, y, B)$ so that $B$ is the graph of some mapping $$ \Phi : u \to y $$ :
	\[
		B = graph(\Phi) = \{(u, \Phi (u)  \mid u \in  \mathbb{U}\}
	.\] 

	\begin{description}
		\item[States:] Let $U$ be a set of inputs and let $y$ be a set of outputs. a response function for a system $(u, y, B)$ is a mapping 
			\[
			 \rho : U x X  \to y 
			.\] 

			such that \[
				B = \{(u, \rho (u, x)  \mid u \in  U, x \in  \mathbb{X} \}
				
			.\]

		X is called a state object. One should think of $X$ as parameterising the outputs for a fixed input. 
			
		Def: a linear general input/output system is a general input/output system $(u, t, B)$ such that 
	\begin{enumerate}
		\item $u and $ y $ are vector spa\subseteqs 
		\item $B  \subsetin U XOR Y$ is a subspace
	\end{enumerate}	
	\end{description}

	One also has linear response functions, meaning that $u$, $y$ and $X$ are vector spaces, and 
	\[
	$ \rho :  \to \mathbb{R} $
	.\] 
\end{document}
